\chapter{Fundamentação teórica}\label{cap:background}

\section{Poses no espaço}

O estudo da cinemática na robótica envolve principalmente o estabelecimento de
sistemas de coordenadas (em inglês, \emph{frames}) para representar posições e
orientações de corpos rígidos, além da caracterização das transformações entre
esses frames. Nesta seção, definiremos uma forma precisa para representar
translações e rotações no espaço por meio de matrizes e vetores. Em seguida,
apresentaremos as transformações homogêneas, que combinam as operações de
rotação e translação em uma única multiplicação matricial, proporcionando uma
maneira concisa de estabelecer a relação entre dois frames distintos.

\subsection{Representado posições e orientações no espaço}

Um sistema de coordenadas ou \emph{frame} é definido por um ponto e um conjunto
de vetores ortonormais que formam uma base no espaço considerado (dois vetores
no plano e três vetores no espaço tridimensional). No contexto da robótica é
muito comum a especificação de uma tarefa com base em coordenadas cartesianas
relativas a algum frame de referência como ilustrado na
figura~\ref{fig:frames}. Podemos tomar o \emph{frame} \(o_0 x_0 y_0 z_0\) como
referência e expressar qualquer ponto \(P\) do espaço através de um vetor com
origem em \(o_0\) e extremidade em \(P\) de modo que escreveríamos:

\begin{equation}
    p^0 = v_0^0 = \begin{bmatrix}
        1 \\
        1 \\
        1
    \end{bmatrix}
\end{equation}

onde o expoente \(0\) indica que o vetor é expresso com coordenadas referentes
ao \emph{frame} \(o_0 x_0 y_0 z_0\). Contudo, ainda observando a
figura~\ref{fig:frames}, podemos expressar o mesmo ponto \(P\) com coordenadas
referentes ao \emph{frame} \(o_1 x_1 y_1 z_1\) através do vetor \(v_1^1\) de
modo que, teríamos por exemplo:

\begin{equation}
    p^1 = v_1^1 = \begin{bmatrix}
        2    \\
        -1.5 \\
        3
    \end{bmatrix}
\end{equation}

Além disso, poderíamos também calcular \(v_1^0\) e \(v_0^1\) de modo que as
coordenadas do vetor dependem do sistema de referência utilizado.

\begin{figure}
    \centering
    \includegraphics[width=0.8\linewidth]{Images/frames.pdf} % Adjust the width as needed
    \caption{Dois sistemas de coordenadas, um ponto P e vetores que o representam.}\label{fig:frames}
\end{figure}

Para estabelecer a relação completa entre os dois \emph{frames} precisamos
também considerar a orientação de um com relação ao outro. Podemos expressar
relação de orientação entre os \emph{frames} \(\{0\}\) e \(\{1\}\) utilizando a
matriz de mudança de base \(R_1^0\) dada por:

\begin{equation}
    R_1^0 = \begin{bmatrix}
        x_1 \cdot x_0 & y_1 \cdot x_0 & z_1 \cdot x_0 \\
        x_1 \cdot y_0 & y_1 \cdot y_0 & z_1 \cdot y_0 \\
        x_1 \cdot z_0 & y_1 \cdot z_0 & z_1 \cdot z_0
    \end{bmatrix}
\end{equation}

onde \((\cdot)\) denota o produto escalar entre dois vetores.

A matriz \(R_1^0\) faz parte de um conjunto de matrizes denominado grupo
especial ortogonal de ordem 3 (\(SO(3)\)) que possui algumas propriedades
interessantes dentre elas a facilidade de se calcular sua inversa:

\begin{equation}
    {(R_1^0)}^{-1} = {(R_1^0)}^{\top} = R_0^1
\end{equation}

Se a relação entre os \emph{frames} é composta apenas por uma rotação de
\(\theta\) radianos em torno de cada eixo ordenado, podemos calcular as
seguintes matrizes de transformação elementares:

\begin{equation}
    R_x(\theta) = \begin{bmatrix}
        1 & 0            & 0             \\
        0 & \cos(\theta) & -\sin(\theta) \\
        0 & \sin(\theta) & \cos(\theta)
    \end{bmatrix}
\end{equation}

\begin{equation}
    R_y(\theta) = \begin{bmatrix}
        \cos(\theta)  & 0 & \sin(\theta) \\
        0             & 1 & 0            \\
        -\sin(\theta) & 0 & \cos(\theta)
    \end{bmatrix}
\end{equation}

\begin{equation}
    R_z(\theta) = \begin{bmatrix}
        \cos(\theta) & -\sin(\theta) & 0 \\
        \sin(\theta) & \cos(\theta)  & 0 \\
        0            & 0             & 1
    \end{bmatrix}
\end{equation}

\subsection{Transformações homogêneas}

Um deslocamento rígido no espaço pode ser representado por uma translação pura
seguida de uma rotação pura. De maneira mais precisa, podemos defini-lo como um
par ordenado \((R, d)\) onde \(R \in SO(3)\) e \(d \in \mathbb{R}^3\). O grupo
de todas as transformações rígidas no espaço tridimensional é denominado grupo
especial euclidiano de ordem 3 (\(SE(3)\)) onde vemos claramente que \(SE(3) =
\mathbb{R}^3 \times SO(3)\).

Seja \(R_1^0\) a matriz de rotação que especifica a orientação do sistema de
coordenadas \(\{1\}\) com relação ao sistema de coordenadas \(\{0\}\) e
\(o_1^0\) o vetor de deslocamento une suas origens. Como visto ne seção
anterior, se \(P\) é um ponto fixado ao sistema de coordenadas \(\{1\}\) com
coordenadas locais dadas por \(p^1\) então, podemos expressar as coordenadas de
\(P\) com relação ao sistema de coordenadas \(\{0\}\) usando o seguinte
deslocamento rígido:

\begin{equation}\label{eq:rigid-motion}
    p^0 = R_1^0 p^1 + o_1^0
\end{equation}

A equação (\ref{eq:rigid-motion}) pode ser reescrita de maneira mais compacta
utilizando-se a seguinte formulação matricial:

\begin{equation}
    \begin{bmatrix}
        p^0 \\
        1
    \end{bmatrix} = \begin{bmatrix}
        R_1^0 & o_1^0 \\
        0     & 1
    \end{bmatrix} \begin{bmatrix}
        p^1 \\
        1
    \end{bmatrix} \ \ , R_1^0 \in SO(3) \ \ , o_1^0 \in \mathbb{R}^3
\end{equation}

onde vemos que os elementos de \(SE(3)\) são matrizes de dimensão \(4 \times
4\) e os vetores \(p^i\) são representados em \emph{coordenadas homogêneas }com
uma dimensão extra. A matriz \(T_1^0\) é denominada matriz de
\emph{transformação homogênea} e utilizando o fato de que \(R_1^0\) é uma
matriz ortogonal, podemos facilmente calcular sua inversa:

\begin{equation}
    {(T_1^0)}^{-1} = \begin{bmatrix}
        R_1^0 & o_1^0 \\
        0     & 1
    \end{bmatrix}^{-1} = \begin{bmatrix}
        {(R_1^0)}^{\top} & -{(R_1^0)}^{\top} o_1^0 \\
        0                & 1
    \end{bmatrix}
\end{equation}

\section{Cinemática Direta}

O problema da cinemática em manipuladores consiste em descrever o movimento do
manipulador sem considerar as forças e torques atuantes sob o mesmo
tratando-se, portanto, de uma descrição puramente geométrica. Neste contexto,
uma pergunta natural que surge é como podemos determinar a posição e orientação
do efetuador final na cadeia cinemática dado um conjunto arbitrário de ângulos
articulados. Este problema é conhecido na robótica como a cinemática direta de
um manipulador e pode ser facilmente resolvido se associarmos a cada corpo
rígido da cadeia um sistema de coordenadas (\emph{frame}), expressando também
as relações entre esses \emph{frames} como transformações homogêneas. A pose do
efetuador final fica determinada através de uma série de multiplicações
matriciais. A colocação de \emph{frames} pode ser feita de maneira sistemática
através da utilização da convenção de Denavit-Hartenberg a qual fornece uma
abordagem concisa para representar a estrutura geométrica do manipulador.

\subsection{Cadeias cinemáticas}

Na robótica, uma cadeia cinemática pode ser definida como uma série de corpos
rígidos, também denominados \emph{elos}, conectados por \emph{juntas} que
permitem um movimento relativo das diferentes partes móveis de um manipulador.
Cadeias cinemáticas formam a base do estudo de manipuladores robóticos e
geralmente são representadas através de um grafo onde seus nós constituem os
elos e as arestas as juntas.

Dependendo da topologia desse grafo, podemos classificar uma cadeia cinemática
de diferentes formas. Numa cadeia serial aberta seu grafo consiste numa árvore
onde cada nó possui apenas um único filho e o nó terminal da árvore usualmente
representa o efetuador final (onde uma pinça ou garra robótica ficaria
acoplada, por exemplo). Outros casos incluem grafos com ramificações (existe
mais de um nó terminal) onde denominamos de cadeia paralela ou quando há a
presença de ciclos onde temos uma cadeia fechada. Neste trabalho, iremos nos
limitar à análise de cadeias abertas, as mais comuns no âmbito de manipuladores
robóticos utilizados em aplicações industriais.

Os tipos de juntas presentes no cadeia também são importantes na definição do
alcance e natureza do movimento do manipulador. As mais simples são as juntas
prismáticas e de revolução, cada uma introduzindo um único grau de liberdade ao
sistema. Juntas prismáticas permitem o movimento translacional ao longo de uma
única direção enquanto que juntas de revolução possibilitam um movimento
rotacional ao redor de um eixo específico.

Ademais das juntas básicas, tipos mais complexos incluem juntas esféricas, que
introduzem dois graus de liberdade (rotação ao redor de dois eixos
perpendiculares) e punhos esféricos, compostos por três juntas revolutas
dispostas ortogonalmente, introduzindo três graus de liberdade ao sistema. Vale
ressaltar que independente da complexidade da junta, a maior parte pode ser
reduzida a uma combinação dos dois tipos mais simples tornando suficiente a
descrição de cadeias cinemáticas por meio de uma combinação de juntas
prismáticas ou de revolução.

Um manipulador robótico com $n$ juntas terá $n + 1$ elos, uma vez que cada
junta conecta exatamente dois elos. Iremos enumerar as juntas de $1$ até $n$ e
elos de $0$ a $n$ sendo que o elo de número $0$ representará a base do
manipulador e elo $n$ seu efetuador final. De acordo com essa convenção a junta
$i$ conecta os elos $i - 1$ ao $i$ e tem sua posição fixa com respeito ao elo
anterior. Quando a junta $i$ é atuada, o elo $i$ se movimenta de modo que a
base permanece fixa independente de qual junta é movimentada.

Iremos associar à $i$-ésima junta uma variável $q_i$ representando no caso de
uma junta de revolução o ângulo de rotação e no caso de uma junta prismática o
deslocamento linear:

\begin{equation}
    q_i =
    \begin{cases}
        \theta_i & \text{se a junta $i$ é de revolução} \\
        d_i      & \text{se a junta $i$ é prismática}   \\
    \end{cases}
\end{equation}

A análise cinemática é feita anexando ao elo $i$ da cadeia o sistema de
coordenadas $o_i x_i y_i z_i$. O \emph{frame} $o_0 x_0 y_0 z_0$ associado à
base do manipulador é denominado \emph{frame} da base, inercial ou do mundo. Se
$A_i(q_i)$ é a matriz de transformação homogênea que fornece a posição e
orientação de $o_i x_i y_i z_i$ com relação a $o_{i-1} x_{i-1} y_{i-1} z_{i-1}$
então podemos dizer que a mesma é função unicamente da variável $q_i$ de modo
que:

\begin{equation}
    A(q_i) = A_i = \begin{bmatrix}
        R^{i-1}_i & o^{i-1}_i \\
        0         & 1
    \end{bmatrix}
\end{equation}

Dessa forma, para $i < j$, a matriz $T_j^i$ dada por:

\begin{equation}
    T_j^i = A_{i+1} \cdots A_j = \begin{bmatrix}
        R^i_j & o^i_j \\
        0     & 1
    \end{bmatrix}
\end{equation}

expressa a orientação $R^i_j$ e posição $o^i_j$ de $o_j x_j y_j z_j$ com
relação a $o_i x_i y_i z_i$. Vale ressaltar que a matriz $R^i_j$ é calculada
através da multiplicação matricial:

\begin{equation}
    R^i_j = R^i_{i+1} \cdots R^{j-1}_j
\end{equation}

e o vetor de posição utilizando-se a seguinte equação recursiva:

\begin{equation}
    o^i_j = o^i_{j-1} + R^i_{j-1}o^{j-1}_j
\end{equation}

O problema da cinemática direta pode ser então formulado como o simples cálculo
da matriz $T^0_n$ que expressa a pose do efetuador final com relação ao
\emph{frame} da base:

\begin{equation}\label{eq:fkine}
    T^0_n = A_1 A_2 \cdots A_n = \begin{bmatrix}
        R^0_n & o^0_n \\
        0     & 1
    \end{bmatrix}
\end{equation}

Tendo em vista a infinidade de possibilidades de se anexar os \emph{frames} em
cade elo, para poder calcular as matrizes $A_i$ de forma mais precisa, vamos
estabelecer uma convenção utilizando para isso os parâmetros introduzidos por
Denavit e Hartenberg.

\subsection{Convenção de Denavit-Hartenberg}

A fixação de frames em cada elo pode ser feita de maneira arbitrária para se
obter as matrizes de transformação $A_i$, permitindo assim o cálculo da
cinemática direta. Contudo, o processo de determinação das mesmas matrizes para
um manipulador com $n$ elos começa a ser tornar cada vez mais complexo a medida
que $n$ cresce. A a convenção de Denavit-Hartenberg consiste numa abordagem
sistemática para a obtenção das matrizes $A_i$ de modo a representar a relação
entre frames consecutivos da forma mais concisa o possível além de propiciar
uma padronização de como pesquisadores descrevem a estrutura cinemática de um
manipulador robótico.

De maneira geral, para especificar a matriz de transformação homogênea seriam
necessários 6 parâmetros: três deslocamentos para a componente de translação e
três ângulos para a rotação. Na convenção de Denavit-Hartenberg, a matriz de
transformação $A_i$ associada ao $i$-ésimo elo é descrita através de apenas 4
parâmetros. Isso é obtido através da introdução de duas restrições na colocação
dos frames em cada elo:

\begin{itemize}
    \item (\textbf{DH1}) $x_i$ intersecta o eixo $z_{i-1}$
    \item (\textbf{DH2}) $x_i$ é perpendicular o eixo $z_{i-1}$
\end{itemize}

adicionar figura dh

O cálculo da matriz $A_i$ fica condicionado à determinação dos parâmetros:
$\theta_i$ (joint angle), $d_i$ (link offset), $a_i$ (link length) e $\alpha_i$
(link twist). Como $A_i$ é função da única variável da junta, três parâmetros
são sempre fixos, dependendo apenas da geometria existente entre os frames,
enquanto o quarto parâmetro é livre: $\theta_i$, no caso de uma junta de
revolução e $d_i$, no caso de uma junta prismática. Em posse dos parâmetros
obtidos para cada par de elos consecutivos, o cálculo da matriz $A_i$ é obtido
através da relação:

\begin{align}
    A_i & = Rot_z(\theta_i) \cdot Trans_z(d_i) \cdot Trans_x(a_i) \cdot Rot_x(\alpha_i)             \\
    A_i & = \begin{bmatrix}
                c_{\theta_i} & -s_{\theta_i} & 0 & 0 \\
                s_{\theta_i} & c_{\theta_i}  & 0 & 0 \\
                0            & 0             & 1 & 0 \\
                0            & 0             & 0 & 1
            \end{bmatrix} \begin{bmatrix}
                              1 & 0 & 0 & 0   \\
                              0 & 1 & 0 & 0   \\
                              0 & 0 & 1 & d_i \\
                              0 & 0 & 0 & 1
                          \end{bmatrix} \notag                                                    \\
        & \phantom{=} \times \ \ \begin{bmatrix}
                                     1 & 0 & 0 & a_i \\
                                     0 & 1 & 0 & 0   \\
                                     0 & 0 & 1 & 0   \\
                                     0 & 0 & 0 & 1
                                 \end{bmatrix} \begin{bmatrix}
                                                   1 & 0            & 0             & 0 \\
                                                   0 & c_{\alpha_i} & -s_{\alpha_i} & 0 \\
                                                   0 & s_{\alpha_i} & c_{\alpha_i}  & 0 \\
                                                   0 & 0            & 0             & 1
                                               \end{bmatrix} \notag                 \\
    A_i & = \begin{bmatrix}
                c_{\theta_i} & -s_{\theta_i}c_{\alpha_i} & s_{\theta_i}s_{\alpha_i}  & a_i c_{\theta_i} \\
                s_{\theta_i} & c_{\theta_i}c_{\alpha_i}  & -c_{\theta_i}s_{\alpha_i} & a_i s_{\theta_i} \\
                0            & s_{\alpha_i}              & c_{\alpha_i}              & d_i              \\
                0            & 0                         & 0                         & 1
            \end{bmatrix} \label{eq:dh-matrix}
\end{align}

onde $c_{\cdot}$ e $s_{\cdot}$ denotam $\cos(\cdot)$ e $\sin(\cdot)$,
respectivamente. A tabela~\ref{tab:dh-parameters} descreve de maneira detalhada
a definição de cada parâmetro de Denavit-Hartenberg.

\begin{table}[htbp]
    \centering
    \begin{tabular}{c c}
        \toprule
        \textbf{Parâmetro} & \textbf{Definição}                                                                                                 \\
        \midrule
        $\theta_i$         & \makecell[l]{O ângulo entre os eixos $\mathbf{x}_i$ e $\mathbf{x}_{i+1}$ em torno do eixo $\mathbf{z}_{i-1}$}      \\
        \midrule
        $d_i$              & \makecell[l]{A distância da origem do sistema de coordenadas $\{i\}$                                               \\ até o eixo $\mathbf{x}_{i+1}$ ao longo do eixo $\mathbf{z}_i$} \\
        \midrule
        $a_i$              & \makecell[l]{A distância entre os eixos $\mathbf{z}_i$ e $\mathbf{z}_{i+1}$ ao longo do eixo $\mathbf{x}_{i+1}$;   \\ para eixos que se intersectam, é paralela a $\mathbf{z}_i \times \mathbf{z}_{i+1}$} \\
        \midrule
        $\alpha_i$         & \makecell[l]{O ângulo entre o eixo $\mathbf{z}_i$ e o eixo $\mathbf{z}_{i+1}$ em torno do eixo $\mathbf{x}_{i+1}$} \\
        \bottomrule
    \end{tabular}
    \caption{Descrição dos parâmetros Denavit-Hartenberg}\label{tab:dh-parameters}
\end{table}

\subsection{Cinemática direta de um braço planar}

Um braço planar é um tipo de manipulador serial cujo espaço de trabalho se
limita a um plano. A figura~\ref{fig:3r-planar-arm} mostra um braço planar do
tipo 3R, o qual possui três elos e três juntas de revolução acoplados em série.
A escolha da colocação do frame da base \(o_0x_0y_0z_0\) é totalmente
arbitrária e ao tomar como indicado na figura (com o eixo \(z\) apontando para
fora do papel) a fixação dos frames subsequentes na cadeia cinemática fica
restrita a convenção de Denavit-Hartenberg adotada. A tabela DH para esse
manipulador é dada por:

\begin{table}[htbp]
    \centering
    \begin{tabular}{c c c c c}
        \toprule
        \textbf{Elo} & \(\theta\)   & \(d\) & \(a\)   & \(\alpha\) \\
        \midrule
        1            & \(\theta_1\) & 0     & \(a_1\) & 0          \\
        2            & \(\theta_2\) & 0     & \(a_2\) & 0          \\
        3            & \(\theta_3\) & 0     & \(a_3\) & 0          \\
        \bottomrule
    \end{tabular}
    \caption{Parâmetros DH para o braço planar 3R da figura~\ref{fig:3r-planar-arm}.}\label{tab:dh-parameters-planar-arm}
\end{table}

Dados os valores fixos \(a_i\) que indicam o comprimento do elo \(i\), as
únicas variáveis livres no cálculo da cinemática direta são os ângulos das
juntas ($q_i = \theta_i$), desse modo, vamos denotar \(\theta_1 + \theta_2 =
\theta_{12}\), \(\cos(\theta_1 + \theta_2) = c_{12}\) e assim por diante. Para
\(i = 1, 2, 3\) as matrizes $A_i$ são calculadas com o auxílio da
equação~\ref{eq:dh-matrix}:

\begin{figure}
    \centering
    \includegraphics[width=0.8\linewidth]{Images/3r-planar.pdf} % Adjust the width as needed
    \caption{Braço planar do tipo 3R.}\label{fig:3r-planar-arm}
\end{figure}

\begin{equation}
    \mathbf{A}_i = \begin{bmatrix}
        c_i & -s_i & 0 & a_i c_i \\
        s_i & c_i  & 0 & a_i s_i \\
        0   & 0    & 1 & 0       \\
        0   & 0    & 0 & 1
    \end{bmatrix}
\end{equation}

Já para as matrizes \(T^0_i\), utilizamos a equação~\ref{eq:fkine}:

\begin{equation}
    \mathbf{T}^0_1 = \mathbf{A}_1
\end{equation}

\begin{equation}
    \mathbf{T}^0_2 = \mathbf{A}_1 \cdot \mathbf{A}_2 = \begin{bmatrix}
        c_{12} & -s_{12} & 0 & a_1c_1 + a_2c_{12} \\
        s_{12} & c_{12}  & 0 & a_1s_1 + a_2s_{12} \\
        0      & 0       & 1 & 0                  \\
        0      & 0       & 0 & 1
    \end{bmatrix}
\end{equation}

\begin{equation}
    \mathbf{T}^0_3 = \mathbf{A}_1 \cdot \mathbf{A}_2 \cdot \mathbf{A}_3 = \begin{bmatrix}
        c_{123} & -s_{123} & 0 & a_1c_1 + a_2c_{12} + a_3c_{123} \\
        s_{123} & c_{123}  & 0 & a_1s_1 + a_2s_{12} + a_3s_{123} \\
        0       & 0        & 1 & 0                               \\
        0       & 0        & 0 & 1
    \end{bmatrix}
\end{equation}

As três primeiras entradas da última coluna da matriz \(T^0_3\) dão a posição
\(\mathbf{P} = {\left[ x \ y \ z \right]}^{\top}\) do efetuador final em função
da configuração do manipulador. Note que $z = 0$ quaisquer que sejam os ângulos
das juntas pois, como esperado, o manipulador é planar. Além disso, analisando
a componente de rotação, fica evidente que a orientação do efetuador final com
relação ao frame da base é dada pela soma dos ângulos das juntas: $\psi =
    \theta_{123}$.

\section{Cinemática Diferencial}

Na seção anterior vimos como podemos estabelecer uma relação entre a
configuração de um manipulador com $n$ juntas com a pose do efetuador final no
espaço SE3. Nesta seção, iremos investigar de que forma se dá a relação de um
vetor de velocidades no espaço das juntas com a velocidade do efetuador final
no espaço de trabalho. Iremos ver que a matriz Jacobiana, atuando como uma
generalização da derivada para o caso multidimensional, é responsável por
estabelecer um mapeamento linear entre as velocidades e tem papel crucial na
caracterização da qualidade do movimento de um manipulador através da análise
das singularidades cinemáticas. Por fim, iremos analisar o problema da
cinemática inversa diferencial, uma aplicação direta do mapeamento estabelecido
pela matriz Jacobiana, proporcionando a geração eficiente de trajetórias
cartesianas (retilíneas) no espaço de trabalho do manipulador.

\subsection{A Jacobiana do Manipulador}

Dado um manipulador com $n$ juntas vamos considerar

\begin{equation}
    T_{n}^{0}(q) = \begin{bmatrix}
        R_{n}^{0}(q) & o_{n}^{0}(q) \\
        0            & 1
    \end{bmatrix}
\end{equation}

a transformação homogênea que expressa a pose do efetuador final com relação ao
\emph{frame} da base que, como já vimos, é função apenas da configuração \(q =
\begin{bmatrix}
    q_1 & \cdots & q_n
\end{bmatrix}^\top\).

Buscamos estabelecer relações da seguinte forma:

\begin{align}
    v_n^0 = J_v \dot{q} \\
    \omega_n^0 = J_\omega \dot{q}
\end{align}

onde \(v_n^0\) e \(\omega_n^0\) expressam respectivamente as velocidades linear
e angular do efetuador final e \(J_v\), \(J_\omega\) são matrizes de dimensão
\(3 \times n\). De maneira mais compacta, podemos escrever

\begin{equation}
    \xi = J \dot{q}
\end{equation}

onde teremos:

\begin{equation}
    \xi = \begin{bmatrix}
        v_n^0 \\
        \omega_n^0
    \end{bmatrix} \text{ e } J = \begin{bmatrix}
        J_v \\
        J_\omega
    \end{bmatrix}
\end{equation}

O vetor \(\xi\) de dimensão \(6 \times 1\) é denominado de velocidade do corpo
rígido (em inglês \emph{body velocity} ou \emph{twist}). Note também que a
matriz \(J\), chamada \emph{jacobiana do manipulador} ou simplesmente
jacobiana, é usualmente uma matriz de dimensão \(6 \times n\).

O cálculo da jacobiana, pode ser feito de maneira simples e sistemática ao
analisarmos as componentes angular e linear separadamente. No primeiro caso,
sabemos que a velocidade angular do efetuador final pode ser obtida através da
soma sucessiva das velocidades angulares de cada elo:

\begin{equation}\label{eq:angular-velocity}
    \omega_n^0 = \omega_{1}^0 + \omega_{2}^0 + \cdots + \omega_n^{0}
\end{equation}

Se a junta \(i\) é prismática não há rotação em torno do eixo \(z_{i-1}\) de
modo que \(\omega_i^{i-1} = 0\). Caso contrário, se a junta \(i\) é de
revolução a rotação dá se em torno do eixo $z_{i-1}$ com magnitude $\dot{q}_i$
de modo que:

\begin{equation}
    \omega_i^{0} = \dot{q_i} z_{i-1}^{0}
\end{equation}

onde obviamente teremos \(z_{0}^{0} = k = \begin{bmatrix}
    0 & 0 & 1
\end{bmatrix}^\top\)

A orientação do eixo \(z_{i}\) com relação ao \emph{frame} da base é dada por
\(z_{i}^0 = R_{i}^0 k\), então substituindo na equação
(\ref{eq:angular-velocity}) podemos escrever:

\begin{equation}
    \omega_n^0 = \rho_1 \dot{q_1} k + \rho_2 \dot{q_2} R_1^0 k + \cdots + \rho_n \dot{q_n} R_{n-1}^0 k
\end{equation}

onde \(\rho_i = 0\) se a junta \(i\) é prismática e \(\rho_i = 1\) caso
contrário. Fica claro então que a metade inferior da jacobiana é dada por:

\begin{equation}
    J_\omega = \begin{bmatrix}
        \rho_1 k & \rho_2 R_1^0 k & \cdots & \rho_n R_{n-1}^0 k
    \end{bmatrix}
\end{equation}

A metade superior da jacobiana é obtida calculando-se o vetor \(\dot{o}_n^0\).
Aplicando a regra da cadeia, temos:

\begin{equation}\label{eq:linear-velocity}
    \dot{o}_n^0 = \frac{\partial o_n^0}{\partial q_1} \dot{q_1} + \frac{\partial o_n^0}{\partial q_2} \dot{q_2} + \cdots + \frac{\partial o_n^0}{\partial q_n} \dot{q_n}
\end{equation}

onde fica claro que a i-ésima coluna de \(J_v\) é dada por:

\begin{equation}
    J_{v_i} = \frac{\partial o_n^0}{\partial q_i}
\end{equation}

Para obter a expressão de \(J_{v_i}\) vamos analisar novamente o caso de juntas
prismáticas e de revolução separadamente. No caso de uma única junta
prismática, então o efetuador final apresenta apenas translação ao longo do
eixo \(z_{i-1}\) de modo que:

\begin{equation}
    \dot{o}_n^0 = \dot{q_i} R_{i-1}^0 \begin{bmatrix}
        0 \\
        0 \\
        1
    \end{bmatrix} = \dot{q_i} z_{i-1}^0
\end{equation}

Comparando com a (\ref{eq:angular-velocity}) vemos que:

\begin{equation}
    J_{v_i} = z_{i-1}^0
\end{equation}

Já para o caso de uma junta de revolução, a velocidade linear do efetuador
final devido ao movimento do elo \(i\) é da forma \(\omega \times r\) dada pela
sua componente tangencial ao círculo de centro no ponto \(o_{i-1}\) e
extremidade no ponto \(o_n\) onde:

\begin{align*}
    \omega & = \dot{q_i} z_{i-1}^0 \\
    r      & = o_n^0 - o_{i-1}^0
\end{align*}

onde finalmente chegamos à expressão:

\begin{equation}
    J_{v_i} = z_{i-1}^0 \times (o_n^0 - o_{i-1}^0)
\end{equation}

Resumindo as equações obtidas acima, podemos calcular a jacobiana de qualquer
manipulador serial utilizando o seguinte procedimento:

A parte superior da matriz jacobiana \(J_v\) será:

\begin{equation}
    J_v = \begin{bmatrix}
        J_{v_1} & J_{v_2} & \cdots & J_{v_n}
    \end{bmatrix}
\end{equation}

onde a i-ésima coluna \(J_{v_i}\) é dada por:

\begin{equation}
    J_{v_i} =
    \begin{cases}
        z_{i-1}^0 \times (o_n^0 - o_{i-1}^0) & \text{se a junta $i$ é de revolução} \\
        z_{i-1}                              & \text{se a junta $i$ é prismática}   \\
    \end{cases}
\end{equation}

Já a parte inferior da matriz jacobiana \(J_\omega\) será:

\begin{equation}
    J_\omega = \begin{bmatrix}
        J_{\omega_1} & J_{\omega_2} & \cdots & J_{\omega_n}
    \end{bmatrix}
\end{equation}

onde a i-ésima coluna \(J_{\omega_i}\) é dada por:

\begin{equation}
    J_{\omega_i} =
    \begin{cases}
        z_{i-1}^0 & \text{se a junta $i$ é de revolução} \\
        0         & \text{se a junta $i$ é prismática}   \\
    \end{cases}
\end{equation}

Note que o cálculo da jacobiana torna-se possível apenas com o conhecimento da
função de cinemática direta \(T_n^0\) mostrando-se uma maneira simples e
sistemática para calcular não so a velocidade do efetuador final mas também a
velocidade de qualquer ponto da estrutura cinemática do manipulador.

Como exemplo, considere o manipulador planar 3R introduzido na seção anterior.
Com base no procedimento descrito acima, a jacobiana do manipulador é dada por:

\begin{equation}
    J(q) = \begin{bmatrix}
        z_0^0 \times (o_3^0 - o_0^0) & z_1^0 \times (o_3^0 - o_1^0) & z_2^0 \times (o_3^0 - o_2^0) \\
        z_0^0                        & z_1^0                        & z_2^0
    \end{bmatrix}
\end{equation}

onde as termos envolvidos são:

\begin{align*}
    o_0^0 = \begin{bmatrix}
                0 \\
                0 \\
                0
            \end{bmatrix} \ \ o_1^0 = \begin{bmatrix}
                                          a_1 c_1 \\
                                          a_1 s_1 \\
                                          0
                                      \end{bmatrix} \ \ o_2^0 & = \begin{bmatrix}
                                                                      a_1 c_1 + a_2 c_{12} \\
                                                                      a_1 s_1 + a_2 s_{12} \\
                                                                      0
                                                                  \end{bmatrix} \ \ o_3^0 = \begin{bmatrix}
                                                                                                a_1 c_1 + a_2 c_{12} + a_3 c_{123} \\
                                                                                                a_1 s_1 + a_2 s_{12} + a_3 s_{123} \\
                                                                                                0
                                                                                            \end{bmatrix} \\
    z_0^0                                     & = z_1^0 = z_2^0 = \begin{bmatrix}
                                                                      0 \\
                                                                      0 \\
                                                                      1
                                                                  \end{bmatrix}
\end{align*}

Desenvolvendo as expressões acima, obtemos:

\begin{equation}
    J(q) = \begin{bmatrix}
        -a_1 s_1 - a_2 s_{12} - a_3 s_{123} & -a_2 s_{12} - a_3 s_{123} & -a_3 s_{123} \\
        a_1 c_1 + a_2 c_{12} + a_3 c_{123}  & a_2 c_{12} + a_3 c_{123}  & a_3 c_{123}  \\
        0                                   & 0                         & 0            \\
        0                                   & 0                         & 0            \\
        0                                   & 0                         & 0            \\
        1                                   & 1                         & 1            \\
    \end{bmatrix}
\end{equation}

Note que o manipulador planar não provoca qualquer translação ao longo do eixo
\(z\) uma vez que qualquer contribuição de \(J_{v_i}\) é nula na terceira
componente. Além disso, a única componente de rotação influenciada pelo
movimento das juntas é a rotação em torno também do eixo do eixo \(z\)
evidenciado pela terceira componente de \(J_{\omega_i}\) igual a 1.

\subsection{Singularidades}

A matriz jacobiana, de dimensão \(6 \times n\), estabelece o mapeamento linear
entre as velocidades das juntas e do efetuador final através da relação:

\begin{equation}\label{eq:jacobian-mapping}
    \xi = J(q) \dot{q}
\end{equation}

que coloca de forma explícita a dependência da configuração atual do
manipulador no cálculo de \(J\). Tal mapeamento implica que qualquer vetor de
velocidades do efetuador final é uma combinação linear das colunas da matriz
jacobiana:

\begin{equation}
    \xi = J_1 \dot{q_1} + J_2 \dot{q_2} + \cdots + J_n \dot{q_n}
\end{equation}

Uma vez que \(\xi \in \mathbb{R}^6\) o manipulador so conseguirá desempenhar
uma velocidade arbitrária se todas as colunas de \(J\) forem linearmente
independentes, ou seja, quando o posto da matriz jacobiana for igual a \(6\).
Para uma matriz \(J \in \mathbb{R}^{6 \times n}\) é sempre verdade que
\(\text{posto}(J) \leq \min(6, n)\). Com efeito, no caso do manipulador planar
3R tínhamos \(n = 3\) e desse modo \(\text{posto}(J) \leq 3\) evidenciando o
fato de que o manipulador não consegue desenvolver qualquer velocidade no
espaço de trabalho.

Configurações paras as quais o posto da matriz jacobiana é menor que o máximo
possível (em inglês \emph{rank deficiency}) são denominadas de
\emph{singularidades} ou configurações singulares. Identificar configurações
singulares é de grande importância para o controle de manipuladores por
diversos motivos, entre eles:

\begin{itemize}
    \item Singularidades representam configurações nas quais a mobilidade do manipulador
          é reduzida, ou seja, não é possível impor um movimento arbitrário ao efetuador
          final.
    \item Quando o manipulador está em uma singularidade, pode haver infinitas soluções
          para o problema de cinemática inversa.
    \item Nas proximidades de uma singularidade, pequenas variações nas velocidades no
          espaço operacional podem causar velocidades ilimitadas no espaço das juntas.
\end{itemize}

Quando a matriz jacobiana é quadrada, podemos usar o fato de que seu
determinante se anula em configurações singulares, contudo ainda assim o
problema de determinar o conjunto de configurações é difícil, pois precisamos
resolver a equação

\begin{equation}
    \det(J(q)) = 0
\end{equation}

que geralmente envolve termos com alto grau de não linearidade. Nas próximas
seções, examinaremos técnicas que viabilizam um esquema de controle capaz de se
afastar de configurações singulares ao explorar a redundância presente em
manipuladores planares. Isso se refere a casos em que a matriz Jacobiana é
retangular, apresentando mais velocidades no espaço das juntas (colunas) do que
velocidades possíveis no espaço de trabalho (linhas).

\subsection{Cinemática Inversa Diferencial}

Se a matriz jacobiana definida na equação (\ref{eq:jacobian-mapping}) é
quadrada e não singular, podemos resolver o problema de cinemática inversa
através da simples inversão da mesma:

\begin{equation}\label{eq:resolved-rate}
    \dot{q} = J^{-1}(q) \xi
\end{equation}

Se a configuração inicial do manipulador \(q(0)\) é conhecida as posições das
juntas podem ser calculadas integrando as velocidades no tempo:

\begin{equation}
    q(t) = q(0) + \int_{0}^{t} \dot{q}(\tau) d\tau
\end{equation}

A integração em tempo discreto pode ser feita utilizando técnicas de métodos
numéricos. A abordagem mais simples consiste na integração pelo método de
Euler, onde as posições das juntas no instante atual \(t_k\) são utilizadas
para calcular a configuração do manipulador no instante posterior \(t_{k+1} =
t_k + \Delta t\):

\begin{align}
    q(t_{k + 1}) & = q(t_k) + \dot{q}(t_k) \delta_t
\end{align}

onde \(\delta_t\) é um intervalo de integração apropriado (tempo de
amostragem).

O esquema de controle descrito acima é conhecido como \emph{resolved rate
    control}, o qual consegue de maneira simples e elegante o solucionar o problema
de gerar movimentos no efetuador final de velocidade constante sem recorrer à
soluções numéricas ou analíticas para o cálculo da cinemática inversa. Tal
esquema é útil na geração de trajetórias retilíneas no espaço de trabalho,
conhecidas como trajetórias cartesianas, uma vez que a componente translacional
da velocidade do efetuador final tem direção constante ao longo de todo o
trajeto pode ser tratada de maneira independente da componente rotacional.

Como exemplo, ainda considerando o manipulador planar 3R, poderíamos apenas
especificar um vetor de velocidades \(\xi\) que leva em conta as componentes do
plano \(xy\) da velocidade linear e a componente de rotação angular em torno do
eixo \(z\) de modo que:

\begin{equation}
    \xi = \begin{bmatrix}
        v_x \\
        v_y \\
        \omega_z
    \end{bmatrix}
\end{equation}

Assim a matriz jacobiana se torna livre das linhas que possuem apenas zeros:

\begin{equation}
    J(q) = \begin{bmatrix}
        -a_1 s_1 - a_2 s_{12} - a_3 s_{123} & -a_2 s_{12} - a_3 s_{123} & -a_3 s_{123} \\
        a_1 c_1 + a_2 c_{12} + a_3 c_{123}  & a_2 c_{12} + a_3 c_{123}  & a_3 c_{123}  \\
        1                                   & 1                         & 1
    \end{bmatrix}
\end{equation}

e contanto que não seja singular pode ser facilmente invertida. Desse modo, se
quisermos por exemplo, gerar um movimento retilíneo no efetuador final paralelo
ao eixo \(x\) do plano \(xy\) com velocidade constante, basta tomar \(\xi = \begin{bmatrix}
    v_x & 0 & 0
\end{bmatrix}^\top\) com \(0 < v_x \ll 1\).

\section{Manipuladores redundantes}

Manipuladores cinematicamente redundantes, são aqueles que possuem mais juntas
do que o número estritamente necessário para a execução de uma determinada
tarefa. Este excedente de juntas confere a esses manipuladores um nível
aumentado de destreza, permitindo-lhes navegar em ambientes complexos com maior
flexibilidade. Nesta seção vamos introduzir uma solução geral para o problema
da cinemática inversa diferencial quando o matriz jacobiana é retangular,
envolvendo o conceito da sua \emph{pseudo-inversa}. Em seguida, utilizando a
decomposição em valores singulares de \(J\) iremos fornecer uma descrição
geométrica e qualitativa da destreza associada à uma dada configuração através
dos conceitos do elipsoide e da medida de manipulabilidade. Por fim, vamos ver
como podemos utilizar a solução geral fornecida pela pseudo-inversa para
otimizar diferentes índices de performance com o objetivo de evitar
singularidades e limites mecânicos das juntas.

\subsection{Pseudo-Inversa da Jacobiana}

Num manipulador cinematicamente redundante, a matriz jacobiana de dimensão \(m
\times n\) será retangular (\(m < n\)). Isso significa que \(J\) possui mais
colunas do que linhas e nesse caso existem infinitas soluções para o problema
de cinemática inversa diferencial. Uma solução possível nesse caso é tratar o
problema como um de otimização, onde buscamos uma solução que minimize por
exemplo \(||\dot{q}||^2\). Pode ser mostrado que tal solução é dada por:

\begin{equation}\label{eq:pseudo-inverse}
    \dot{q} = J^\dag \xi + (I_n - J^\dag J) \dot{q_0}
\end{equation}

onde \(\dot{q_0}\) é um vetor de velocidades arbitrário e a matriz \(J^\dag\) é
conhecida como \emph{matriz inversa de Moore-Penrose} ou apenas
\emph{pseudo-inversa} de \(J\) e é dada por:

\begin{equation}
    J^\dag = J^\top {(J J^\top)}^{-1}
\end{equation}

Vale notar que o termo \(I_n - J^\dag J\) atua projetando o vetor \(\dot{q_0}\)
no espaço nulo de \(J\). Com efeito, aplicando a jacobiana à esquerda na
equação (\ref{eq:pseudo-inverse}) temos:

\begin{align*}
    J \dot{q} & = J J^\dag \xi + J (I_n - J^\dag J) \dot{q_0}                                   \\
              & = J J^\top {(J J^\top)}^{-1} \xi + (J - J J^\top {(J J^\top)}^{-1} J) \dot{q_0} \\
              & = \xi + (J - J) \dot{q_0}                                                       \\
    J \dot{q} & = \xi
\end{align*}

permitindo que o manipulador realize movimentos internos no espaço das juntas
que que não afetam a velocidade \(\xi\) do efetuador final.

\subsection{Medida de Manipulabilidade}

Uma maneira de investigar mais a fundo o mapeamento linear estabelecido pela
jacobiana é entender como a mesma ``deforma'' o vetor \(\dot{q}\) de entradas
para produzir o vetor \(\xi\) de saídas. Para isso, podemos considerar o disco
formado pelo conjunto de velocidades com norma unitária:

\begin{equation}
    \left\Vert \dot{q} \right\Vert^2 = q_1^2 + q_2^2 + \cdots + q_n^2 \leq 1
\end{equation}

Substituindo a solução de menor norma \(\dot{q} = J^\dag \xi\):

\begin{align}\label{eq:manipulability-ellipsoid}
    \left\Vert \dot{q} \right\Vert^2 & = \dot{q}^\top \dot{q} \notag                                                    \\
                                     & = {(J^\dag \xi)}^\top J^\dag \xi \notag                                          \\
                                     & = \xi^\top {(J^\top {(J J^\top)}^{-1})}^\top J^\top {(J J^\top)}^{-1} \xi \notag \\
                                     & = \xi^\top {(J J^\top)}^{-1} (J J^\top) {(J J^\top)}^{-1} \xi \notag             \\
    \left\Vert \dot{q} \right\Vert^2 & = \xi^\top {(J J^\top)}^{-1} \xi \leq 1
\end{align}

A equação (\ref{eq:manipulability-ellipsoid}) define uma região no espaço de
trabalho conhecido como \emph{elipsoide de manipulabilidade} que representa
todas as velocidades possíveis do efetuador final para uma dada configuração do
manipulador. Esse fato pode ser facilmente verificado ao considerarmos a
decomposição em valores singulares (SVD) da jacobiana \(J = U \Sigma V^\top\):

\begin{align}\label{eq:manipulability-ellipsoid-svd}
    \left\Vert \dot{q} \right\Vert^2 & = \xi^\top {(U \Sigma V^\top V \Sigma^\top U^\top)}^{-1} \xi \notag \\
                                     & = \xi^\top {(U \Sigma^2 U^\top)}^{-1} \xi \notag                    \\
                                     & = \xi^\top U \Sigma^{-2} U^\top \xi \notag                          \\
    \left\Vert \dot{q} \right\Vert^2 & = {(U^\top \xi)}^\top \Sigma^{-2} (U^\top \xi)
\end{align}

onde sabemos que \(U\) e \(V\) são matrizes ortogonais, isto é \(U^{-1} =
U^\top\) e \(V^{-1} = V^\top\). Além disso a matriz

\begin{equation}
    \Sigma^{-2} = \begin{bmatrix}
        \sigma_1^{-2} &               &        &               & \\
                      & \sigma_2^{-2} &        &               & \\
                      &               & \ddots &               & \\
                      &               &        & \sigma_m^{-2} & \\
    \end{bmatrix}
\end{equation}

é diagonal e os termos que \(\sigma_1 \geq \sigma_2 \geq \cdots \geq \sigma_m\) são os valores
singulares de \(J\). Por fim, ao fazermos a substituição \(w = U^\top \xi\) podemos reescrever a equação
(\ref{eq:manipulability-ellipsoid-svd}) como:

\begin{equation}
    w^\top \Sigma^{-2} w = \sum_{i=1}^m{\frac{{w_i}^2}{{\sigma_i}^2}} \leq 1
\end{equation}

evidenciado que o disco é mapeado num elipsoide com eixos alinhados a um
sistema de coordenadas rotacionado por \(U^\top\). No sistema de coordenadas
original, os semi-eixos do elipsoide são dados pelos vetores \(\sigma_i u_i\).

A medida de manipulabilidade \(\mu\) é definida como o produto dos valores
singulares de \(J\):

\begin{equation}\label{eq:manipulability}
    \mu = \sigma_1 \sigma_2 \cdots \sigma_m
\end{equation}

que é proporcional ao volume do elipsoide de manipulabilidade. Ao passo que nos
aproximamos de uma singularidade, um ou mais dos valores singulares de \(J\) se
aproximam de zero, reduzindo o volume do elipsoide e consequentemente a
destreza do manipulador. Isso pode ser visualizado na
figura~\ref{fig:manipulability-ellipsoid} onde para o braço planar 3R, o
elipsoide de manipulabilidade é mostrado para diferentes configurações do
manipulador e vai se tornando cada vez mais achatado à medida que nos
aproximamos do limite do espaço de trabalho.

\begin{figure}
    \centering
    \includegraphics[width=0.9\textwidth]{Images/3r-ellipsoid.png}
    \caption{Elipsoide de manipulabilidade para diferentes configurações do manipulador planar 3R.}\label{fig:manipulability-ellipsoid}
\end{figure}

\subsection{Resolução de Redundância}

Ao estabelecermos a solução geral dada pela equação (\ref{eq:pseudo-inverse}),
dissemos que o vetor \(\dot{q_0}\) pode ser escolhido arbitrariamente. Uma
possibilidade é tomá-lo de forma a maximar algum índice de performance \(w\),
para isso escolhendo o vetor na direção do gradiente:

\begin{equation}
    \dot{q_0} = k_0 {\left( \frac{\partial w(q)}{\partial q} \right)}^\top
\end{equation}

onde \(k_0 > 0\) é uma constante positiva que determina o tamanho do passo.

Uma escolha natural para o índice de performance \(w\) é a \emph{medida de
    manipulabilidade de Yoshikawa}:

\begin{equation}
    w(q) = \sqrt{\det(J(q){J(q)}^\top)}
\end{equation}

onde vale ressaltar que é equivalente àquela definida na a equação
(\ref{eq:manipulability}) uma vez que se \(\lambda_i\) são os autovalores de
\(J J^\top\) então \(\sigma_i = \sqrt{\lambda_i}\).

Apesar do cálculo da manipulabilidade ser relativamente simples, o mesmo não se
pode dizer para o cálculo do seu gradiente devido à complexidade das expressões
envolvidas. Uma alternativa é utilizar uma métrica mais simples como a
\emph{distância para os limites mecânicos das juntas} dada por:

\begin{equation}
    w(q) = -\frac{1}{2n} \sum_{i=1}^{n}{{\left(\frac{q_i - \bar{q_i}}{q_{iM} - q_{im}}\right)}^2}
\end{equation}

Ao maximar tal índice, espera-se que o manipulador mantenha-se próximo ao ponto
central de atuação de cada junta, evitando assim configurações singulares no
limite do espaço de trabalho. Além disso, o vetor gradiente pode ser calculado
de maneira analítica onde cada coordenada é dada por:

\begin{equation}
    \frac{\partial w(q)}{\partial q_i} = -\frac{1}{n} \frac{q_i - \bar{q_i}}{{(q_{iM} - q_{im})}^2}
\end{equation}

Outro ponto a salientar é que a escolha do tamanho do passo \(k_0\) é crucial
para a performance do algoritmo. Se \(k_0\) for muito pequeno o processo de
otimização pode se tornar muito lento, enquanto que se \(k_0\) for muito grande
isso pode levar até mesmo uma diminuição ou até mesmo não convergência do valor
de \(w\) devido a oscilações em torno do ponto de máximo local.

No próximo capítulo, iremos utilizar os conceitos apresentados até agora na
concepção de um simulador minimalista do manipulador planar 3R denominado
\emph{Snakesim}. O objetivo principal será explorar a resolução de redundância
presente na execução de trajetórias cartesianas no plano \(xy\) do espaço de
trabalho para maximizar os índices tanto de manipulabilidade quanto de distância
para os limites das juntas introduzidos nessa seção.
